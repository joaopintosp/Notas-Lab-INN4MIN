%! Author = joaos
%! Date = 15/11/2024
\pagestyle{fancy}

\section*{Lixiviação com Tiossulfato de sódio pentahidratado}

\newday{15 Novembro 2024}

Hoje realizou-se a lixiviação do restante material da amostra \texttt{A08.3}, a mesma utilizada na digestão ácida -~\nameref{day:8-novembro-2024}.

A lixiviação será realizada com \TSP{}, \SCP{} e \AMO{}.
As concentrações dos reagentes são as seguintes:
\begin{itemize}
    \item \tsp{} = 1~M\@;
    \item \scp{} = 0,01~M\@;
    \item \amo{} = 2~M\@.
\end{itemize}

A lixiviação será feita com uma razão sólido/líquido de 1 para 2 (S/L = 1/2).

A lixiviação foi efetuada a temperatura ambiente, com uma agitação de 450~rpm, durante 8~horas.
Foi utilizado um reator de borossilicato de 1~L\@.

As quantidades de reagentes utilizados foram as seguintes:
\begin{itemize}
    \item $\mathrm{m}_{\left[ \tsp \right]} = \SI{99,268}{g}$
    \item $\mathrm{m}_{\left[ \scp \right]} = 0,9987 \approx \SI{1}{g}$
    \item $\mathrm{V}_{\left[ \amo \right]} = \SI{15,138}{mL}$
\end{itemize}

\begin{marginfigure}[-4\baselineskip]
    \centering
    \includegraphics[width=\linewidth]{figures/reagentes-lixiviação 1}
    \caption{Reagentes utilizados na lixiviação.}
    \label{fig:reagentes-lixiviacao-1}
\end{marginfigure}

Num balão volumétrico de 250~mL foi colocado 15~mL de \AMO{}, com uma pipeta de 10~mL\@.
O volume restante foi preenchido com água destilada, até perfazer os 250~mL do balão.
Num gobelé, foi medida a massa de \TSP{} (99,34~g) e a massa de \SCP{} (1,01~g) que vão ser utilizados.
De seguida, juntou-se os conteúdos do balão volumétrico (água + \AMO{}) ao gobelé com o \TSP{} e \SCP{}, agitou-se com uma vareta de vidro até estar bem dissolvido e homogeneizado.

Foi medida a massa de minério a ser lixiviada (200,00~g), da amostra \texttt{A08.3}.
O minério foi colocado dentro do reator de borossilicato.
Adicionou-se ao reator a solução com os reagentes dissolvidos e acrescentou-se 150~mL de água de forma a perfazer os 400~mL de fase líquida, respeitando a relação S/L\@.

\begin{marginfigure}[-8\baselineskip]
    \centering
    \includegraphics[width=0.9\linewidth]{figures/lixiviação-1 a decorrer}
    \caption{Lixiviação a decorrer.}
    \label{fig:lixiviacao1-a-decorrer}
\end{marginfigure}

Regulou-se o agitador e definiu-se uma velocidade de rotação de 450~rpm.
Deixou-se a trabalhar durante cerca de 5~minutos.
De seguida, parou-se o agitador, deixou-se decantar um pouco e mediu-se o pH (7,25) e o Eh (-86,0~mV - valor medido; 133~mV - valor convertido).

\marginnote{Os valores de pH e de Eh não são fidedignos. Os equipamentos de medição não estavam calibrados, portanto os valores medidos podem não ser os valores reais.}

Uma vez registados os valores de pH e Eh, retomou-se o funcionamento do agitador.

Deixou-se a lixiviar durante 8~horas.
No fim da lixiviação, mediu-se novamente o pH (6,15) e o Eh (-60,5~mV - valor medido; 159~mV - valor convertido).

De seguida, montou-se o sistema de filtragem, composto por um filtro de Büchner, um Kitasato, uma bomba de vácuo e um papel de filtro.
Filtrou-se e mediu-se o volume do licor de lixiviação - 369~mL\@.

\begin{marginfigure}
    \centering
    \includegraphics[width=0.9\linewidth]{figures/licor e agua lavagem lix tiossulfato}
    \caption{Licor de lixiviação e água de lavagem (Tiossulfato).}
    \label{fig:licor-lix-agua-lavagem-tiossulfato}
\end{marginfigure}

O resíduo sólido que foi filtrado foi lavado com 500~mL de água.
Colocou-se o resíduo de novo no reator, adicionou-se 500~mL de água e ligou-se o agitador, deixando lavar durante 30~minutos.

Após os 30~minutos, filtrou-se novamente o material.
Mediu-se o volume de solução de água de lavagem - 490~mL\@.

Tanto o licor de lixiviação como a água de lavagem, foram colocados em recipientes de vidro, identificados e armazenados - Figura~\ref{fig:licor-lix-agua-lavagem-tiossulfato}.

\begin{marginfigure}
    \centering
    \includegraphics[width=0.9\linewidth]{figures/residuo_sol_lix_tiossulfato}
    \caption{Resíduo sólido da lixiviação (Tiossulfato).}
    \label{fig:res-solido-lix-tiossulfato}
\end{marginfigure}

O material, já lavado e filtrado (Figura~\ref{fig:res-solido-lix-tiossulfato}), foi colocado numa estufa a secar durante o fim-de-semana.
Após estar seco, mediu-se a massa - 192,61~g (já a descontar a massa do vidro de relógio e do papel de filtro).
Reservou-se.

\hrulefill

%%%%%%%%%%%%%%%%%%%%%%%%%%%%%%%%%%%%%%%%%%%%%%%%%%%%%%%%

\newday{19 Novembro 2024}\label{day:19-novembro-2024}

Hoje preparou-se o resíduo sólido da lixiviação para se realizar a digestão ácida.
O resíduo sólido proveniente da lixiviação com \TSP{}, foi moído no moinho de anéis.

Foram retirados 50~g de material, com o divisor de amostras\sidenote{Ver Figura~\ref{fig:divisor_de_amostras_retsch}.} tendo sido colocado 10~g em cinco copos distintos\sidenote{Ver procedimento do dia~\nameref{day:7-novembro-2024}.}.
Esses copos foram colocados na mufla e aquecidos até 700~\graus{} e deixados a arrefecer durante a noite.


\hrulefill

\newpage

\newday{20 Novembro 2024}

Hoje realizou-se a digestão ácida do resíduo de lixiviação com \TSP{}, dando seguimento ao dia~\nameref{day:19-novembro-2024}.

Foi seguido o mesmo procedimento do dia~\nameref{day:8-novembro-2024}.
\marginnote{Ver o procedimento em detalhe para mais informação.}

Utilizou-se a mesma relação de ácido sulfúrico e ácido nítrico, 1:3.
Ou seja, 6~mL de \ce{HNO3} e 18~mL de \ce{HCl}.
Foram feitos 3 ataques, com 1~hora e 30~minutos entre cada um.
Totalizando em 4~horas e 30~minutos de digestão.

No fim dos 3~ataques, juntou-se 4~mL de \ce{HCl} a cada um dos copos e verteu-se para balões volumétricos de 50~ml, separando o resíduo sólido do líquido.
O volume restante dos balões foi preenchido com água destilada.
O resíduo sólido foi colocado numa placa de Petri e reservado.

Posteriormente, o conteúdo dos balões volumétricos será filtrado para ser analisado na absorção atómica.

\hrulefill