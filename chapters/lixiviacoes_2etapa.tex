\newday{Digestão ácida de um novo saco de material}

Nesta segunda etapa de lixiviações irá ser utilizado um novo saco de material, pois o material restante do saco \texttt{A08} foi utilizado nos ensaios preliminares.

Para isso, é necessário selecionar um saco aleatório e analisar a teor em ouro desta nova ``alimentação''.

\marginnote{Uma das digestões ácidas partiu-se e portanto apenas foi possível analisar 4 das 5 digestões efetuadas.
	No entanto, como os valores obtidos são bastante próximos, considerou-se que a média dos teores em \ce{Au} é representativa.}

Selecionou-se o saco \texttt{A04.3} e procedeu-se à digestão ácida do material, seguindo o procedimento do dia~\nameref{day:7-novembro-2024} e do dia~\nameref{day:8-novembro-2024}.

Após a digestão ácida, procedeu-se à análise na absorção atómica.
Da qual se obteve a seguinte tabela referente às concentrações em \ce{Au} do saco \texttt{A04.3}.

\begin{table*}[!ht]
	\centering
	\begin{tabular}{@{}lccccc@{}}
		\toprule
		\textbf{Amostra} & \textbf{Conc. (mg/L)} & \textbf{Qt. metal na sol. (mg)} & \textbf{Teor \ce{Au} (mg/g)} & \textbf{Teor \ce{Au} (\%)} & \textbf{Teor \ce{Au} (ppm)} \\ \midrule
		\textbf{Dig. 1} & 1,91 & 0,0955 & 0,0096 & 0,0010 & 9,550 \\
		\textbf{Dig. 2} & 1,85 & 0,0925 & 0,0093 & 0,0009 & 9,250 \\
		\textbf{Dig. 3} & 1,99 & 0,0995 & 0,0100 & 0,0010 & 9,950 \\
		\textbf{Dig. 4} & 1,77 & 0,0885 & 0,0089 & 0,0009 & 8,850 \\
		\bottomrule
	\end{tabular}
\end{table*}

\newpara

Sendo que a média dos teores em \ce{Au} é \textbf{de 9,40~ppm}.

\hrulefill

\newday{4 Dezembro 2024}\label{day:4-dezembro-2024}

Hoje realizou-se o \textbf{segundo ensaio de lixiviação com Tioureia}.
Utilizou-se o material restante do saco \texttt{A08.4}, o mesmo material utilizado na lixiviação com Tioureia do dia~\nameref{day:22-novembro-2024}.

Como dito anteriormente, suspeita-se que os resultados não satisfatórios obtidos no ensaio preliminar com Tioureia sejam consequência de uma quantidade de reagentes insuficiente.
Nesse sentido, neste segundo ensaio, colocou-se o dobro da concentração de cada um dos reagentes para que se possa verificar se esse foi o problema.

Portanto, as concentrações utilizadas foram as seguintes:

\begin{marginfigure}[2.5cm]
	\centering
	\includegraphics[width=0.9\textwidth]{figures/Reagentes Tioureia.JPG}
	\caption{Reagentes utilizados na lixiviação com Tioureia, ensaio 2.}
	\label{fig:reagentes-tioureia2}
\end{marginfigure}

\begin{itemize}
	\item[-] Tioureia, \tioureia{} - 200~g/kg de minério;
	\item[-] Sulfato de ferro (III) penta-hidratado, \sulfe{}: 0,5~g/kg de minério;
	\item[-] Ácido sulfúrico, \acsul{} - o necessário para obter uma solução com pH = 1.
\end{itemize}

Utilizou-se 100~g de material restante do saco \texttt{A08.4}, para a lixiviação.
Foi utilizada uma relação sólido-líquido de 1 para 5 (S/L = 1/5), portanto para 100~g de sólido teremos 500~mL de líquido.

A lixiviação foi efetuada a uma temperatura ambiente, com uma agitação de 350~rpm, durante 6~horas.
Foi utilizado um reator de borossilicato de 500~mL.

\marginnote{As alterações feitas em relação ao
	ensaio preliminar foram a temperatura (de 60~\graus{} para temperatura ambiente), a concentração de tioureia (de 100~g/kg para 200~g/kg) e a concentração de sulfato de ferro (III) penta-hidratado (de 0,5~g/kg para 1~g/kg).}

A quantidade de reagentes adicionada foi a seguinte:
\begin{itemize}
	\item[-] $\mathrm{m}_{\left[\tioureia{}\right]}$ = 20~g;
	\item[-] $\mathrm{m}_{\left[\sulfe{}\right]}$ = 0,123~g;
	\item[-] $\mathrm{V}_{\left[\acsul{}\right]}$ = 1,7~mL para ajustar pH = 1.
\end{itemize}

No reator de borossilicato, foi adicionado 500~mL de água destilada e 20~g de \tioureia{}. 
Dissolveu-se a \tioureia{} e ajustou-se o pH, adicionando 1,7~mL de \acsul{}, que resultou num pH = 1,251.
Após acidificar a solução, adicionou-se \sulfe{} e o minério no reator, regulando o agitador para 350~rpm e dando início à lixiviação às 10h40.

Após 5~minutos de lixiviação, interrompeu-se a agitação e mediu-se o pH (1,315). 
Foi-se verificando o pH ao longo da lixiviação, para garantir que a solução se encontrava com pH = 1, adicionando \acsul{} sempre que necessário.

Adicionou-se, durante as 6~horas de lixiviação, 0,6~mL de \acsul{}. No total, adicionou-se 2,3~mL de \acsul{}, sendo que 1,7~mL foram adicionados antes da lixiviação e 0,6~mL durante a lixiviação para manter o pH = 1.

Foi preparada uma solução de lavagem com água destilada acidificada (pH = 1,099). 
A relação S/L para lavagem foi de 1/1,5, ou seja, 150~mL de solução de lavagem para 100~g de minério.

Uma vez terminada a lixiviação, filtrou-se a o material com um sistema de filtragem composto por um filtro de Büchner, um balão Kitasato, papel de filtro e bomba de vácuo.
Filtrou-se a solução, mediu-se o volume de licor filtrado (495~mL) e mediu-se o pH do licor (1,801).

\begin{marginfigure}
	\centering
	\includegraphics[width=0.9\textwidth]{figures/Licores de lixiviação - tioureia ensaio 2}
	\caption{Licor de lixiviação e águas de lavagem da lixiviação com Tioureia, ensaio 2.}
	\label{fig:licores-lixiviacao-tioureia-ensaio2}
\end{marginfigure}

Irão ser efetuadas duas lavagens, portanto colocou-se o sólido filtrado novamente no reator de borossilicato e adicionou-se 150~mL de solução de lavagem.
Deixou-se a lavar durante 30~minutos, com agitação de 350~rpm.
Uma vez terminada a lavagem, filtrou-se novamente com o mesmo sistema de filtragem.
Mediu-se o volume da água de lavagem (150~mL) e mediu-se o pH da água de lavagem (pH = 1,232).

\marginnote{Ocorreu novamente precipitação de sólido nos frascos de armazenamento do licor de lixiviação e das águas de lavagem. Ver Figura~\ref{fig:precipitado-lix-tioureia}.}

Colocou-se o sólido filtrado novamente no reator de borossilicato e adicionou-se 150~mL de solução de lavagem, para efetuar a segunda lavagem.
Deixou-se a lavar durante 30~minutos, com agitação de 350~rpm.
Uma vez terminada a lavagem, filtrou-se novamente com o mesmo sistema de filtragem.
Mediu-se o volume da água de lavagem (150~mL) e mediu-se o pH da água de lavagem (pH = 1,216).

% \begin{marginfigure}
%     \centering
%     \includegraphics[width=0.9\linewidth]{figures/Resíduo Sólido Lixiviação Tioureia Ensaio 2.jpg}
%     \caption{Resíduo sólido seco da lixiviação com Tioureia, ensaio 2.}
%     \label{fig:residuo-solido-tioureia-ensaio2}
% \end{marginfigure}

Tanto o licor de lixiviação como as águas de lavagem foram guardados em frascos devidamente identificados para posterior análise.

O sólido filtrado foi colocado numa placa de Petri e deixado a secar na estufa a 60~\graus{} durante 10~horas.

\begin{marginfigure}[\baselineskip]
	\centering
	\includegraphics[width=0.9\textwidth]{figures/Precipitado Toureia}
	\caption{Precipitado no licor de lixiviação com Tioureia, ensaio 2.}
	\label{fig:precipitado-lix-tioureia}
\end{marginfigure}

Após estar seco, mediu-se a massa do resíduo de lixiviação, subtraindo as massas dos papéis de filtro - 89,49~g.

O resíduo sólido será submetido a digestão ácida para análise posterior na absorção atómica.

\newthought{Concentrações em \ce{Au}:} Licor de lixiviação - 1,088~mg/L; Água de lavagem 1 - 0,64~mg/L; Água de lavagem 2 - 0,13~mg/L; resíduo (média) - 6,90~ppm.

\hrulefill

\newday{Digestões ácidas dos resíduos de lixiviação - Tioureia, ensaio 2}

Hoje realizou-se as digestões ácidas do resíduo sólido proveniente da lixiviação com Tioureia, ensaio 2, do dia~\nameref{day:4-dezembro-2024}.

O procedimento seguido para a preparação da amostra para ser colocada na mufla e da digestão ácida em si foi os do dia~\nameref{day:7-novembro-2024} e \nameref{day:8-novembro-2024}, respetivamente.

Após a análise na absorção atómica, obteve-se a Tabela~\ref{tab:aas-concentracao-au-res-tioureia2} referente às concentrações em \ce{Au} no resíduo de lixiviação com Tioureia, ensaio 2.

\begin{table}[!ht]
	\centering
	\begin{tabularx}{\textwidth}{@{}lCCC@{}}
		\toprule
		\textbf{Amostra} & \textbf{Absorção} & \textbf{Conc. (mg/L)} & \textbf{Teor \ce{Au} (ppm)} \\ \midrule
		\textbf{Dig. 1}  & 0,02813           & 1,01                  & 5,0587                      \\
		\textbf{Dig. 2}  & 0,04310           & 1,54                  & 7,7224                      \\
		\textbf{Dig. 3}  & 0,04126           & 1,48                  & 7,3950                      \\
		\textbf{Dig. 4}  & 0,04119           & 1,48                  & 7,3826                      \\
		\textbf{Dig. 5}  & 0,03875           & 1,39                  & 6,9484                      \\ \bottomrule
	\end{tabularx}
	\caption{Concentração em \ce{Au} no resíduo de lixiviação com Tioureia, ensaio 2.}
	\label{tab:aas-concentracao-au-res-tioureia2}
\end{table}

Da absorção atómica também se determinou a concentração em \ce{Au} do licor de lixiviação (\textbf{1,088~mg/L}), da água de lavagem 1 (\textbf{0,64~mg/L}) e da água de lavagem 2 (\textbf{0,13~mg/L}).

\hrulefill

\newday{11 Dezembro 2024}\label{day:11-dezembro-2024}

Hoje realizou-se a segunda lixiviação com Citrato de Sódio di-hidratado, \citratodi{}. Utilizou-se o material que sobrou do ensaio preliminar, realizado no dia~\nameref{day:29-novembro-2024}, o saco \texttt{A08.2}.

Utilizou-se os mesmos reagentes: citrato de sódio di-hidratado, \citratodi{}; tiossulfato de sódio penta-hidratado \tsp{}; sulfato de cobre penta-hidratado, \sulfcu{}; hidróxido de sódio, \hidso{}.
\marginnote{As alterações feitas em relação ao ensaio preliminar foram a concentração do tiossulfato de sódio penta-hidratado (de 0,2~M para 0,1~M) e a temperatura de lixiviação (de 90~\graus{} para 70~\graus{}).}
As concentrações de cada um dos reagentes foi a seguinte:

\begin{itemize}
	\item[-] \citratodi{} = 0,2~M;
	\item[-] \tsp{} = 0,1~M;
	\item[-] \sulfcu{} = 0,1~M;
	\item[-] \hidso{} = 5~M e 15~M.
\end{itemize}

A lixiviação foi realizada com uma relação sólido-líquido de 1 para 5 (S/L = 1/5) e foi utilizado 100~g de minério.

A lixiviação foi efetuada a 90~\graus{}, com uma agitação de 400~rpm, durante 9~horas.

\begin{marginfigure}
	\centering
	\includegraphics[width=0.9\linewidth]{figures/reagentes lixiviação citrato}
	\caption{Reagentes utilizados na segunda lixiviação (citrato).}
	\label{fig:reagentes-lix-citrato2}
\end{marginfigure}

As quantidades de reagentes adicionadas foram as seguintes:

\begin{itemize}
	\item[-] $\mathrm{m}_{\left[\citratodi{}\right]}$ = 21,41~g;
	\item[-] $\mathrm{m}_{\left[\tsp{}\right]}$ = 12,41~g;
	\item[-] $\mathrm{m}_{\left[\sulfcu{}\right]}$ = 12,48~g.
\end{itemize}

Num gobelé, foi medida a massa de citrato de sódio di-hidratado (21,41~g), a massa de tiossulfato de sódio penta-hidratado (12,41~g) e a massa de sulfato de cobre penta-hidratado (12,48) que vão ser utilizados.
Num balão volumétrico de 500~mL foi medido o volume de água necessário para respeitar a relação sólido-líquido.

Colocou-se, no reator de borossilicato, os 500~mL de água destilada e juntou-se os reagentes sólidos.
Colocou-se o agitador mecânico, regulado a 400~rpm e dissolveu-se os reagentes na água destilada, antes de se colocar o minério.

Mediu-se o pH da solução de lixiviação, sem o minério (pH = 2,895).
Adicionou-se uma solução\sidenote{Na verdade, foram adicionadas duas soluções com diferentes concentrações. Primeiramente uma solução de \hidso{} a 5~M, mas como o pH continuava fora do intervalo desejado após se adicionar uma quantidade considerável, trocou-se para uma solução a 15~M, para prevenir alterações na relação sólido-líquido.} de \hidso{}, para que o pH da solução de lixiviação estivesse dentro do intervalo [7; 11].
O pH medido da solução de lixiviação sem o minério, após a adição de 3,4~mL de \hidso{} a 5~M e 2,15~mL de \hidso{} a 15~M, foi de 11,85.

Uma vez ajustado o pH, adicionou-se as 100~g de minério à solução de lixiviação e deu-se início à lixiviação, pelas 10h30.
Deixou-se lixiviar durante 5~minutos e mediu-se novamente o pH (5,8).
Ajustou-se novamente o pH, pela adição 1,85~mL de \hidso{} a 15~M, até que o pH fosse de 7,246.

Foi-se medindo o pH de hora em hora. 
O valor de pH manteve-se constante, pelo que não foi necessário adicionar mais \hidso{} durante a lixiviação.

No fim das 9~horas de lixiviação, filtrou-se a solução com um sistema de filtragem composto por um filtro de Büchner, um balão Kitasato, papel de filtro e bomba de vácuo.
Filtrou-se e mediu-se o volume de licor de lixiviação (487~mL). 
Mediu-se o pH (6,719) e guardou-se o licor de lixiviação num frasco devidamente identificado.

O bolo de minério foi lavado com 150~mL de água destilada. 
Foi utilizada uma relação de lavagem de 1 para 1,5 (S/L = 1/1,5) e efetuaram-se duas lavagens, de 30~minutos cada.

Após a primeira lavagem, filtrou-se o material e mediu-se o volume da água da 1ª lavagem (112~mL) e o pH (pH = 6,970) \sidenote{O volume de água de lavagem filtrada é muito inferior aos 150~mL de água de lavagem adicionada, porque o filtro de papel colmatou e não se conseguiu filtrar os 38~mL de água restante. Colocou-se o bolo com a água de novo no reator e procedeu-se à segunda lavagem, adicionando na mesma os 150~mL de água destilada.}.
O volume de água filtrada da 2ª lavagem foi de 156~mL e o pH medido foi de 7,676.
As águas de lavagem foram devidamente identificadas e colocadas em recipientes de vidro.

O material, já lavado e filtrado, foi colocado numa placa de Petri e deixado a secar na estufa a 60~\graus{} durante 10~horas.

Após estar seco, mediu-se a massa do resíduo de lixiviação, subtraindo as massas dos papéis de filtro - 90,29~g.

O resíduo sólido seco foi desagrado no moinho de anéis, e colocado na mufla para se proceder à digestão ácida.

\newthought{Concentrações em \ce{Au}:} Licor de lixiviação - 1,34~mg/L; Água de lavagem 1 - 0,31~mg/L; Água de lavagem 2 - 0,30~mg/L; resíduo (média) - 6,67~ppm.

\hrulefill

\newday{Digestões ácidas dos resíduos de lixiviação - Citrato, ensaio 2}

Realizou-se a digestão ácida do resíduo sólido proveniente da lixiviação com citrato de sódio di-hidratado, ensaio 2, do dia~\nameref{day:11-dezembro-2024}.
O procedimento seguido foi o mesmo que o do dia~\nameref{day:7-novembro-2024} e \nameref{day:8-novembro-2024}.

Após a digestão ácida, procedeu-se à análise na absorção atómica.
Após essa análise, obteve-se a Tabela~\ref{tab:aas-concentracao-au-res-citrato2} referente às concentrações em \ce{Au} no resíduo de lixiviação com citrato de sódio di-hidratado, ensaio 2.

\newpage

\begin{table}[!ht]
	\centering
	\begin{tabularx}{\textwidth}{@{}lCCC@{}}
		\toprule
		\textbf{Amostra} & \textbf{Absorção} & \textbf{Conc. (mg/L)} & \textbf{Teor \ce{Au} (ppm)} \\ \midrule
		\textbf{Dig. 1}  & 0,04577           & 1,32                  & 6,5853                      \\
		\textbf{Dig. 2}  & 0,04538           & 1,31                  & 6,5289                      \\
		\textbf{Dig. 3}  & 0,04684           & 1,35                  & 6,7399                      \\
		\textbf{Dig. 4}  & 0,04634           & 1,33                  & 6,6676                      \\
		\textbf{Dig. 5}  & 0,04762           & 1,37                  & 6,8526                      \\ \bottomrule
	\end{tabularx}
	\caption{Concentração em \ce{Au} no resíduo de lixiviação com Citrato, ensaio 2.}
	\label{tab:aas-concentracao-au-res-citrato2}
\end{table}

Da absorção atómica também se determinou a concentração em \ce{Au} do licor de lixiviação (\textbf{1,34~mg/L}), da água de lavagem 1 (\textbf{0,31~mg/L}) e da água de lavagem 2 (\textbf{0,30~mg/L}).

\hrulefill

\newday{9 Janeiro 2025}\label{day:9-janeiro-2025}

Hoje realizou-se o \textbf{segundo ensaio de lixiviação com Tiossulfato}.
Utilizou-se minério de um novo saco, o saco \texttt{A04.3}, que já foi submetido a digestão ácida e analisado na \emph{AAS}, apresentando um teor (médio) de \textbf{9,40~ppm}.
Este saco é a nova alimentação que será utilizada para os restantes ensaios nesta segunda etapa de lixiviações.

\marginnote{As alterações relativamente ao ensaio preliminar foram a concentração de \tsp{} (de 1~M para 0,1~M), a concentração de \amo{} (de 2~M para 1~M), a temperatura (de ambiente para 50~\graus{}), a quantidade de minério (de 200~g para 100~g), o tempo de lixiviação (de 8~horas para 7~horas), o pH (de \textpm~7 para \textpm~11), a relação sólido-líquido (de 1/2 para 1/4) e a utilização de \hidso{} para ajustar o pH.}

Utilizou-se os mesmos reagentes que no ensaio preliminar, realizado no dia~\nameref{day:15-novembro-2024}: tiossulfato de sódio penta-hidratado, \tsp{}; sulfato de cobre penta-hidratado, \sulfcu{}; amónia a 25~\%, \amo{}.
As concentrações de cada um dos reagentes foi a seguinte:

\begin{itemize}
	\item[-] \tsp{} = 0,1~M;
	\item[-] \sulfcu{} = 0,01~M;
	\item[-] \amo{} = 1~M.
\end{itemize}

A lixiviação foi realizada com uma relação sólido-líquido de 1 para 4 (S/L = 1/4) e foi utilizado 100~g de minério.

A lixiviação foi efetuada a 50~\graus{}, com uma agitação de 400~rpm, durante 7~horas.
Foi utilizado um reator de borossilicato de 500~mL.

\begin{marginfigure}[2cm]
	\centering
	\includegraphics[width=0.9\textwidth]{figures/Lixiviação com Tiossulfato - ensaio 2}
	\caption{Lixiviação com Tiossulfato a decorrer, ensaio 2.}
	\label{fig:lix-tiossulfato2}
\end{marginfigure}

As quantidades de reagentes adicionadas foram as seguintes:

\begin{itemize}
	\item[-] $\mathrm{m}_{\left[\tsp{}\right]}$ = 9,99265 $\approx$ 10~g;
	\item[-] $\mathrm{m}_{\left[\sulfcu{}\right]}$ = 0,09976 $\approx$ 1~g;
	\item[-] $\mathrm{V}_{\left[\amo{}\right]}$ = 15,58~mL.
\end{itemize}

Numa bureta de 500~mL, colocou-se cerca de 200~mL de água destilada e juntou-se 15,38~mL de amónia a 25~\%.
Perfez-se o volume restante com água destilada, obtendo-se uma solução de 400~mL, respeitando a relação sólido-líquido.
Juntou-se à solução preparada os reagentes sólidos (\tsp{} e \sulfcu{}) e dissolveu-se os reagentes na água destilada, antes de se colocar o minério.

As 100~g de minério foram colocadas dentro do reator de borossilicato, juntou-se a solução de lixiviação e regulou-se o agitador mecânico para uma velocidade de rotação de 400~rpm, dando inicio à lixiviação às 10h20.
Passado 5~minutos, mediu-se o pH (9,3) e ajustou-se\sidenote{Adicionou-se 45 gotas de uma solução de \hidso a 15~M, com uma pipeta \emph{pasteur}. Cada gota tem aproximadamente 0,05~mL, ou seja, adicionou-se cerca de 2,0~mL de \hidso{} a 15~M para ajustar o pH a \textpm~11.} com \hidso{} a 15~M, até que o pH fosse de 11,002.
Uma vez ajustado o pH, deixou-se a lixiviar durante 7~horas, medindo-se o pH de hora a hora.
Não foi necessário ajustar o pH durante a lixiviação.

No fim das 7~horas de lixiviação, filtrou-se a solução com um sistema de filtragem composto por um filtro de Büchner, um balão Kitasato, papel de filtro e bomba de vácuo.
O volume de licor de lixiviação filtrado foi de 380~mL.
O pH do licor de lixiviação foi de 11.172.
O licor de lixiviação foi guardado num frasco devidamente identificado.

O bolo de minério foi lavado com 150~mL de água destilada (S/L = 1/1,5), durante 30~minutos.
Após a lavagem, mediu-se o volume da água de lavagem 1 (142,5~mL) e o pH (10,990).

O bolo foi lavado uma segunda vez com 150~mL de água destilada durante 30~minutos.
Após a lavagem, mediu-se o volume da água de lavagem 2 (149~mL) e o pH (10,828).

\begin{marginfigure}
	\centering
	\includegraphics[width=0.9\textwidth]{figures/Licores de lixiviação - tiossulfato 2}
	\caption{Licor de lixiviação e águas de lavagem da lixiviação com Tiossulfato, ensaio 2.}
	\label{fig:licores-lix-tiossulfato2}
\end{marginfigure}

As águas de lavagem foram guardadas em recipientes de vidro devidamente identificados.

O material, já lavado e filtrado, foi colocado numa placa de Petri e deixado a secar na estufa a 60~\graus{} durante 10~horas.
Após estar seco mediu-se a massa do resíduo sólido de lixiviação, subtraindo as massas dos papéis de filtro - 90,50~g.

O resíduo sólido será submetido a digestão ácida para análise posterior na absorção atómica.

\newthought{Concentrações em \ce{Au}:} Licor de lixiviação - 1,24~mg/L; Água de lavagem 1 - 0,408~mg/L; Água de lavagem 2 - 0~mg/L; resíduo (média) - 18,21~ppm.

\hrulefill
\newpage

\newday{Digestões ácidas dos resíduos de lixiviação - Tiossulfato, ensaio 2}

Realizou-se a digestão ácida do resíduo sólido proveniente da lixiviação com tiossulfato de sódio, ensaio 2, do dia~\nameref{day:9-janeiro-2025}.
O procedimento seguido foi o mesmo que o do dia~\nameref{day:7-novembro-2024} e \nameref{day:8-novembro-2024}.

Após a digestão ácida, procedeu-se à análise na absorção atómica.
Após essa análise, obteve-se a Tabela~\ref{tab:aas-concentracao-au-res-tiossulfato2} referente às concentrações em \ce{Au} no resíduo de lixiviação com tiossulfato de sódio, ensaio 2.

\begin{table}[!ht]
	\centering
	\begin{tabularx}{\textwidth}{@{}lCCC@{}}
		\toprule
		\textbf{Amostra} & \textbf{Absorção} & \textbf{Conc. (mg/L)} & \textbf{Teor \ce{Au} (ppm)} \\ \midrule
		\textbf{Dig. 1} & 0,11173 & 3,638 & 18,189 \\
		\textbf{Dig. 2} & 0,05323 & 1,897 & 9,484 \\
		\textbf{Dig. 3} & 0,06083 & 2,123 & 10,615 \\
		\textbf{Dig. 4} & 0,21233 & 6,632 & 33,159 \\
		\textbf{Dig. 5} & 0,12139 & 3,925 & 19,626 \\ \bottomrule
	\end{tabularx}
	\caption{Concentração em \ce{Au} no resíduo de lixiviação com Tiossulfato, ensaio 2.}
	\label{tab:aas-concentracao-au-res-tiossulfato2}
\end{table}

Da absorção atómica também se determinou a concentração em \ce{Au} do licor de lixiviação (\textbf{1,24~mg/L}), da água de lavagem 1 (\textbf{0,408~mg/L}) e da água de lavagem 2 (\textbf{0~mg/L}).

\hrulefill

\newday{10 Janeiro 2025}\label{day:10-janeiro-2025}

Hoje \textbf{refez-se o ensaio preliminar de lixiviação com Brometo de sódio}, \bromo{}, realizado no dia~\nameref{day:27-novembro-2024}.
O material utilizado foi do saco \texttt{A04.1}.

Utilizou-se os mesmos reagentes que no ensaio preliminar: brometo de sódio, \bromo{}; ácido clorídrico a 37~\%, \acl{}; hipoclorito de sódio 10-15~\%, \hipso{}.
As concentrações de cada um dos reagentes foi a mesma que no ensaio preliminar:

\begin{itemize}
	\item[-] \bromo{} = 0,43~M ou 44~g/L;
	\item[-] \acl{} = 0,42~M;
	\item[-] \hipso{} = 0,23~M.
\end{itemize}

Utilizou-se 200~g de minério da amostra \texttt{A04.1} e a lixiviação foi efetuada com uma relação sólido-líquido de 1 para 1 (S/L = 1/1).
A lixiviação foi efetuada a 60~\graus{}, com uma agitação de 350~rpm, durante 6~horas.

\marginnote[-1cm]{Repetiu-se este ensaio para confirmar os resultados obtidos no ensaio preliminar. Pode ter escapado bromo do reator durante a lixiviação, o que pode ter sido o fator que influenciou o mau rendimento do ensaio preliminar.}

O procedimento seguido para a lixiviação foi exatamente o mesmo protocolo seguido para o dia~\nameref{day:27-novembro-2024}.
Nesse sentido, não se descreverá novamente o procedimento.

No fim da lixiviação, filtrou-se a solução com um sistema de filtragem composto por um filtro de Büchner, um balão Kitasato, papel de filtro e bomba de vácuo.
O volume de licor de lixiviação filtrado foi de 149~mL e o pH foi de 1,260.
O licor de lixiviação foi guardado num frasco devidamente identificado.

\begin{marginfigure}[\baselineskip]
	\centering
	\includegraphics[width=0.9\textwidth]{figures/Lixiviação Bromo Refeito}
	\caption{Lixiviação com Brometo de sódio a decorrer.}
	\label{fig:lix-bromo2}
\end{marginfigure}

Lavou-se o bolo de minério com 300~mL de água destilada (S/L = 1/1,5), durante 30~minutos.
Após a lavagem, filtrou-se o material e mediu-se o volume da água de lavagem (287~mL) e o pH (2,145).

Lavou-se o bolo de minério uma segunda vez com 300~mL de água destilada durante 30~minutos.
Após a lavagem, filtrou-se o material e mediu-se o volume da água de lavagem (295~mL) e o pH (2,558).

\begin{marginfigure}[0.5cm]
	\centering
	\includegraphics[width=0.9\textwidth]{figures/Lixiviação Bromo Refeito de perto}
	\caption{Cor da solução de lixiviação com Bromo.}
	\label{fig:cor-lix-bromo2}
\end{marginfigure}

O material, já lavado e filtrado, foi colocado numa placa de Petri e deixado a secar na estufa a 60~\graus{} durante 10~horas.
Após estar seco, mediu-se a massa do resíduo sólido de lixiviação, subtraindo as massas dos papéis de filtro - 185,84~g.

O resíduo sólido será submetido a digestão ácida para análise posterior na absorção atómica.

\newthought{Concentrações em \ce{Au}:} Licor de lixiviação - 3,811~mg/L; Água de lavagem 1 - 0,779~mg/L; Água de lavagem 2 - 0,347~mg/L; resíduo (média) - 10,18~ppm.

\hrulefill

\newday{Digestões ácidas dos resíduos de lixiviação - Bromo, ensaio 1 refeito}

Realizou-se a digestão ácida do resíduo sólido proveniente da lixiviação com brometo de sódio (\textbf{ensaio preliminar refeito}), do dia~\nameref{day:10-janeiro-2025}.

Após a digestão ácida, procedeu-se à análise na absorção atómica.
Após essa análise, obteve-se a Tabela~\ref{tab:aas-concentracao-au-res-bromo-refeito} referente às concentrações em \ce{Au} no resíduo de lixiviação com brometo de sódio, \textbf{ensaio preliminar refeito}.

\begin{table}[!ht]
	\centering
	\begin{tabularx}{\textwidth}{@{}lCCC@{}}
		\toprule
		\textbf{Amostra} & \textbf{Absorção} & \textbf{Conc. (mg/L)} & \textbf{Teor \ce{Au} (ppm)} \\ \midrule
		\textbf{Dig. 1} & - & 2,139 & 10,695 \\
		\textbf{Dig. 2} & - & 2,024 & 10,120 \\
		\textbf{Dig. 3} & - & 1,934 & 9,670 \\
		\textbf{Dig. 4} & - & 2,015 & 10,075 \\
		\textbf{Dig. 5} & - & 2,068 & 10,340 \\ \bottomrule
	\end{tabularx}
	\caption{Concentração em \ce{Au} no resíduo de lixiviação com Bromo, ensaio preliminar refeito.}
	\label{tab:aas-concentracao-au-res-bromo-refeito}
\end{table}

Da absorção atómica também se determinou a concentração em \ce{Au} do licor de lixiviação (\textbf{3,811~mg/L}), da água de lavagem 1 (\textbf{0,779~mg/L}) e da água de lavagem 2 (\textbf{0,347~mg/L}).

% \hrulefill

\newday{14 Janeiro 2025}\label{day:14-janeiro-2025}

Hoje realizou-se o \textbf{segundo ensaio\sidenote{Atenção que tecnicamente este é o terceiro ensaio realizado com bromo, visto que se repetiu o ensaio preliminar. No entanto, consideremos este como o segundo ensaio realizado.} de lixiviação com Brometo de sódio}, \bromo{}. 

Para a elaboração do procedimento desta lixiviação foi tomado em consideração o artigo ``\emph{Extraction of gold from refractory gold ore using bromate and ferric chloride solution}''~\cite{bromo2_2019}.
A partir deste artigo foram definidas as concentrações dos reagentes a usar e o procedimento a ser tomado para a lixiviação.

No artigo original foi usado Bromato de Potássio, \bromopot{}.
Iremos utilizar Brometo de Sódio, \bromo{}.
Será necessário ter atenção à alteração dos reagentes na preparação da solução de lixiviação.
Nesse sentido, as concentrações de cada um dos reagentes foi a seguinte:

\begin{itemize}
	\item[-] Bromato de Potássio, \bromopot{} = 0,25~M;
	\item[-] Ácido Clorídrico a 37~\%, \acl{} = 0,4~M;
	\item[-] Cloreto férrico, \cloretofe{} = 0,08~M.
\end{itemize}

Utilizou-se 100~g de minério da amostra \texttt{A04.2}.
A lixiviação foi realizada com uma razão sólido-líquido de 1 para 5 (S/L = 1/5).

A lixiviação foi efetuada a temperatura ambiente, com uma agitação de 350~rpm, durante 30~minutos.
Foi utilizado um reator de borossilicato de 500~mL.

Foram realizados os cálculos das quantidades de reagentes necessários, tendo em consideração que se vai utilizar \bromo{} em vez de \bromopot{}.
Dessa forma, a quantidade de reagentes utilizada foi a seguinte:

\marginnote{Caso fosse utilizado o \bromopot{}, a quantidade a adicionar seria 20,88~g, para a concentração especificada.}

\begin{itemize}
	\item[-] $\mathrm{m}_{\left[\bromo{}\right]}$ = 12,86~g;
	\item[-] $\mathrm{m}_{\left[\cloretofe{}\right]}$ = 6,49~g;
	\item[-] $\mathrm{V}_{\left[\acl{}\right]}$ = 16,70~mL;
\end{itemize}

Adicionou-se as 100~g de minério ao reator de borossilicato e juntou-se os reagentes sólidos. 
Montou-se o reator de borossilicato, com 4 aberturas. 
Numa das aberturas, colocou-se a coluna de condensação.
As outras duas aberturas foram fechadas com rolhas de silicone e a última abertura foi utilizada para o agitador mecânico\sidenote{Ainda se selou a abertura do agitador com fita adesiva de papel, para prevenir qualquer tipo de fuga de bromo, permitindo na mesma a rotação do agitador.}.

Num balão volumétrico de 500~mL foi adicionado um pouco de água destilada e juntou-se 16~70~mL de \acl{} a 37~\%. Perfez-se o volume restante com água destilada, obtendo-se uma solução de 500~mL, respeitando a relação sólido-líquido.

Adicionou-se a solução de lixiviação ao reator de borossilicato, selou-se todas as aberturas do reator e deu-se início à lixiviação, às 14h10.
Decidiu-se não medir o pH durante a lixiviação para evitar a fuga de bromo. 
Apenas se mediu o pH após se filtrar a solução de lixiviação.

\marginnote{Não se verificou qualquer tipo de cor avermelhada, como se verificou na lixiviação preliminar com bromo - ver Figura~\ref{fig:cor-lix-bromo2}.}

Passado 30~minutos, filtrou-se a solução com um sistema de filtragem composto por um filtro de Büchner, um balão Kitasato, papel de filtro e bomba de vácuo.
O volume de licor de lixiviação filtrado foi de 480~mL e o pH medido foi de 0,643.

Lavou-se o bolo de minério com 300~mL de água destilada (S/L = 1/3), durante 30~minutos.
Após a lavagem, filtrou-se o material e mediu-se o volume da água de lavagem (295~mL) e o pH (1,635).

O material, já lavado e filtrado, foi colocado numa placa de Petri e deixado a secar na estufa a 60~\graus{} durante 10~horas.
Após estar seco, mediu-se a massa do resíduo sólido de lixiviação - 92,14~g.

O resíduo sólido será submetido a digestão ácida para análise posterior na absorção atómica.

\newthought{Concentrações em \ce{Au}:} Licor de lixiviação - 1,978~mg/L; Água de lavagem - 0,353~mg/L; resíduo - 7,88~ppm.

% \hrulefill

\newday{Digestões ácidas dos resíduos de lixiviação - Bromo, ensaio 2}

Realizou-se a digestão ácida do resíduo sólido proveniente da lixiviação com bromo, ensaio 2, do dia~\nameref{day:14-janeiro-2025}.

Após a digestão ácida, procedeu-se à análise na absorção atómica.
Após essa análise, obteve-se a Tabela~\ref{tab:aas-concentracao-au-res-bromo2} referente às concentrações em \ce{Au} no resíduo de lixiviação com bromo, ensaio 2.

\begin{table}[!ht]
	\centering
	\begin{tabularx}{\textwidth}{@{}lCCC@{}}
		\toprule
		\textbf{Amostra} & \textbf{Absorção} & \textbf{Conc. (mg/L)} & \textbf{Teor \ce{Au} (ppm)} \\ \midrule
		\textbf{Dig. 1} & 0,03522 & 1,569 & 7,846 \\
		\textbf{Dig. 2} & 0,03315 & 1,500 & 7,500 \\
		\textbf{Dig. 3} & 0,03767 & 1,651 & 8,256 \\
		\textbf{Dig. 4} & 0,03679 & 1,622 & 8,1087 \\
		\textbf{Dig. 5} & 0,03433 & 1,539 & 7,6973 \\ \bottomrule
	\end{tabularx}
	\caption{Concentração em \ce{Au} no resíduo de lixiviação com Bromo, ensaio 2.}
	\label{tab:aas-concentracao-au-res-bromo2}
\end{table}

Da absorção atómica também se determinou a concentração em \ce{Au} do licor de lixiviação (\textbf{1,978~mg/L}) e da água de lavagem (\textbf{0,353~mg/L}).

\hrulefill

\newpage

\section*{Resultados da segunda etapa de lixiviações}

\subsection*{Tioureia, ensaio 2}

Após analisar o licor de lixiviação, a água de lavagem e as digestões ácidas dos resíduos sólidos da lixiviação na absorção atómica, obteve-se os seguintes resultados para a segunda lixiviação realizada com Tioureia, do dia~\nameref{day:4-dezembro-2024}.

\begin{table}[!ht]
    \centering
    \begin{tabularx}{\textwidth}{@{}CCCC@{}}
        \toprule
        \textbf{Teor Alim. (ppm)} & \textbf{Rendimento fase líquida} & \textbf{Rendimento fase sólida} & \textbf{Teor Resíduo (ppm)} \\ \midrule
         9,87 & 66,27~\% & 30,08~\% & 6,90 \\ \bottomrule                  
    \end{tabularx}
    \caption{Teor da alimentação original (Tioureia, ensaio 2).}
    \label{tab:original-grade-feed-tioureia-2}
\end{table}

Realizou-se também a compatibilização do teor da alimentação, tendo em conta os valores do teor em \ce{Au} obtidos pelas análises do licor, da água de lavagem e do resíduo sólido da lixiviação.

\begin{table}[!ht]
    \centering
    \begin{tabularx}{\textwidth}{@{}CC@{}}
        \toprule
        \textbf{Teor Alim. compatibilizado (ppm)} & \textbf{Rendimento fase líquida} \\ \midrule
        13,44 & 48,66~\% \\ \bottomrule                  
    \end{tabularx}
    \caption{Teor da alimentação compatibilizado (Tioureia, ensaio 2).}
    \label{tab:compatibalized-grade-feed-tioureia-2}
\end{table}

Ao comparar com os resultados obtidos no ensaio preliminar (Tabela~\ref{tab:original-grade-feed-tioureia} e \ref{tab:compatibalized-grade-feed-tioureia}), é possível verificar que o rendimento da fase líquida foi superior. 

Portanto, ao alterar a temperatura para 60~\graus{}, aumentar a concentração de tioureia para 200~g/kg e a concentração de sulfato de ferro (III) para 1~g/kg, conseguiu-se obter um rendimento superior na fase líquida.

\hrulefill

\subsection*{Citrato, ensaio 2}

Após analisar o licor de lixiviação, a água de lavagem e as digestões ácidas dos resíduos sólidos da lixiviação na absorção atómica, obteve-se os seguintes resultados para a segunda lixiviação realizada com Citrato, no dia~\nameref{day:11-dezembro-2024}.

\begin{table}[!ht]
    \centering
    \begin{tabularx}{\textwidth}{@{}CCCC@{}}
        \toprule
        \textbf{Teor Alim. (ppm)} & \textbf{Rendimento fase líquida} & \textbf{Rendimento fase sólida} & \textbf{Teor Resíduo (ppm)} \\ \midrule
         9,87 & 70,11~\% & 32,37~\% & 6,67 \\ \bottomrule                  
    \end{tabularx}
    \caption{Teor da alimentação original (Citrato, ensaio 2).}
    \label{tab:original-grade-feed-citrato-2}
\end{table}

Realizou-se também a compatibilização do teor da alimentação, tendo em conta os valores do teor em \ce{Au} obtidos pelas análises do licor, da água de lavagem e do resíduo sólido da lixiviação.

\begin{table}[!ht]
    \centering
    \begin{tabularx}{\textwidth}{@{}CC@{}}
        \toprule
        \textbf{Teor Alim. compatibilizado (ppm)} & \textbf{Rendimento fase líquida} \\ \midrule
        13,59 & 50,90~\% \\ \bottomrule                  
    \end{tabularx}
    \caption{Teor da alimentação compatibilizado (Citrato, ensaio 2).}
    \label{tab:compatibalized-grade-feed-citrato-2}
\end{table}

Ao comparar com os resultados obtidos no ensaio preliminar (Tabela~\ref{tab:original-grade-feed-citrato} e \ref{tab:compatibalized-grade-feed-citrato}), é possível verificar que o rendimento da fase líquida foi consideravelmente superior (70,11~\%). 
No entanto, ao compatibilizar o teor da alimentação, o rendimento da fase líquida foi similar ao obtido no ensaio preliminar.

\hrulefill

\subsection*{Tiossulfato, ensaio 2}

Após analisar o licor de lixiviação, a água de lavagem e as digestões ácidas dos resíduos sólidos da lixiviação na absorção atómica, obteve-se os seguintes resultados para a segunda lixiviação realizada com Tiossulfato, no dia~\nameref{day:9-janeiro-2025}.

\begin{table}[!ht]
    \centering
    \begin{tabularx}{\textwidth}{@{}CCCC@{}}
        \toprule
        \textbf{Teor Alim. (ppm)} & \textbf{Rendimento fase líquida} & \textbf{Rendimento fase sólida} & \textbf{Teor Resíduo (ppm)} \\ \midrule
         9,40 & 56,33~\% & 32,37~\% & 18,21 \\ \bottomrule                  
    \end{tabularx}
    \caption{Teor da alimentação original (Tiossulfato, ensaio 2).}
    \label{tab:original-grade-feed-tiossulfato-2}
\end{table}

Realizou-se também a compatibilização do teor da alimentação, tendo em conta os valores do teor em \ce{Au} obtidos pelas análises do licor, da água de lavagem e do resíduo sólido da lixiviação.

\begin{table}[!ht]
    \centering
    \begin{tabularx}{\textwidth}{@{}CC@{}}
        \toprule
        \textbf{Teor Alim. compatibilizado (ppm)} & \textbf{Rendimento fase líquida} \\ \midrule
        23,51 & 22,52~\% \\ \bottomrule                  
    \end{tabularx}
    \caption{Teor da alimentação compatibilizado (Tiossulfato, ensaio 2).}
    \label{tab:compatibalized-grade-feed-tiossulfato-2}
\end{table}

% TODO: Colocar aqui a análise dos resultados do segundo ensaio da tioureia.

\hrulefill

\subsection*{Bromo, ensaio 2}

Após analisar o licor de lixiviação, a água de lavagem e as digestões ácidas dos resíduos sólidos da lixiviação na absorção atómica, obteve-se os seguintes resultados para a segunda lixiviação realizada com Bromo, no dia~\nameref{day:10-janeiro-2025}.

\newpage

\begin{table}[!ht]
    \centering
    \begin{tabularx}{\textwidth}{@{}CCCC@{}}
        \toprule
        \textbf{Teor Alim. (ppm)} & \textbf{Rendimento fase líquida} & \textbf{Rendimento fase sólida} & \textbf{Teor Resíduo (ppm)} \\ \midrule
         9,40 & 112,05~\% & 16,18~\% & 7,882 \\ \bottomrule                  
    \end{tabularx}
    \caption{Teor da alimentação original (Bromo, ensaio 2).}
    \label{tab:original-grade-feed-bromo-2}
\end{table}

Realizou-se também a compatibilização do teor da alimentação, tendo em conta os valores do teor em \ce{Au} obtidos pelas análises do licor, da água de lavagem e do resíduo sólido da lixiviação.

\begin{table}[!ht]
    \centering
    \begin{tabularx}{\textwidth}{@{}CC@{}}
        \toprule
        \textbf{Teor Alim. compatibilizado (ppm)} & \textbf{Rendimento fase líquida} \\ \midrule
        18,42 & 57,21~\% \\ \bottomrule
    \end{tabularx}
    \caption{Teor da alimentação compatibilizado (Bromo, ensaio 2).}
    \label{tab:compatibalized-grade-feed-bromo-2}
\end{table}

% TODO: Colocar aqui a análise dos resultados do segundo ensaio do bromo.

\hrulefill
