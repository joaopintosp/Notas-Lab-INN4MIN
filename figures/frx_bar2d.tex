\begin{figure}[!htb]
    \centering
    \begin{tikzpicture}[font=\footnotesize]
    \centering
      \begin{axis}[
            ybar, axis on top,
            title={Elementos mais presentes nas amostras},
            height=7cm, width=11cm,
            bar width=0.175cm, % Reduzi o tamanho das barras
            axis lines=left,
            x axis line style={opacity=0},
            ymin=0, ymax=6,
            enlarge x limits=true, % Espaçamento proporcional entre os grupos
            ylabel={\% Elemento},
            ytick = {0,1,2,3,4,5,6},
            yticklabel={\pgfmathprintnumber{\tick}~\%},
            symbolic x coords={A01.1, A02.3, A03.2, A04.3, A05.3, A06.4, A07.4, A08.3},
            xtick=data,
            legend style={
                at={(0.5,-0.15)},
                anchor=north,
                legend columns=-1,
                /tikz/every even column/.append style={column sep=0.5cm}
            },
            % Adicionar rótulos às barras (rotacionados para melhor espaçamento)
            nodes near coords,
            every node near coord/.append style={
                font=\scriptsize, rotate=90, anchor=west
            }
        ]

        % Adicionando as barras
        \addplot+[draw=none, color=customblue]
            coordinates {(A01.1,4.5818) (A02.3,4.9766) (A03.2,4.4801) (A04.3,4.8707) (A05.3,4.6077) (A06.4,4.2681) (A07.4,4.6648) (A08.3,4.3410)};
        \addlegendentry{Fe};

        \addplot+[draw=none, color=customorange]
            coordinates {(A01.1,1.9157) (A02.3,2.0522) (A03.2,1.8442) (A04.3,2.0483) (A05.3,1.9080) (A06.4,1.7179) (A07.4,1.9466) (A08.3,1.7770)};
        \addlegendentry{As};

        \addplot+[draw=none, color=customcinza]
            coordinates {(A01.1,1.5312) (A02.3,1.7211) (A03.2,1.3313) (A04.3,1.6292) (A05.3,1.4163) (A06.4,1.5042) (A07.4,1.9466) (A08.3,1.8327)};
        \addlegendentry{K};

        \addplot+[draw=none, color=customyellow]
            coordinates {(A01.1,0.2109) (A02.3,0.2631) (A03.2,0.2261) (A04.3,0.2470) (A05.3,0.2508) (A06.4,0.2246) (A07.4,0.2695) (A08.3,0.2317)};
        \addlegendentry{Ti};

      \end{axis}
    \end{tikzpicture}
    \caption{FRX - Elementos mais presentes nas amostras.}
    \label{fig:frx-elementos-mais-presentes}
\end{figure}